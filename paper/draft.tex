\documentclass[letterpaper,twocolumn,10pt]{article}
\usepackage{usenix2019_v3}

\usepackage{soul}
\usepackage{xspace}
\usepackage{color}
\newcommand{\eg}{{\em e.g.}, }
\newcommand{\ie}{{\em i.e.}, }
\newcommand{\etc}{{\em etc.}\xspace}
\newcommand{\vs}{{\em vs.} }

\newcommand{\cf}[1]{(\emph{Cf}.\S\ref{#1})}
\newcommand{\sx}[1]{(\S\ref{#1})}
\newcommand{\sys}{{\scshape Lya}\xspace}
\newcommand{\toy}{{\tt lya.js}\xspace}
%-------------------------------------------------------------------------------
\begin{document}
%-------------------------------------------------------------------------------

%don't want date printed
\date{}

% make title bold and 14 pt font (Latex default is non-bold, 16 pt)
\title{\Large \bf Module-level Dynamic Analysis}

%for single author (just remove % characters)
\author{
{\rm Grigoris Ntousakis}\\
Technical University of Crete
\and
{\rm Nikos Vasilakis}\\
Massachusetts Institute of Technology
}

\maketitle

\begin{abstract}
We introduce \emph{module-level dynamic analysis}, a new dynamic analysis abstraction that supports dynamic fracture and recombination at the level of individual modules.
By leveraging the ubiquity of third-party modules in today's applications, this analysis provides 
Our implementation, \sys, simplifies the analysis 

\end{abstract}

\section{Introduction}

Some citations to ensure we can see them~\cite{Christophe:2015:DAU:2819009.2819180, Keil:2013:EDA:2508168.2508176, Lehmann:2019:WFD:3297858.3304068, Sun:2018:EDA:3178372.3179527}.


% \section*{Acknowledgments}
% 
% \section*{Availability}

\bibliographystyle{plain}
\bibliography{bib}

\end{document}

%%  LocalWords:  endnotes includegraphics fread ptr nobj noindent
%%  LocalWords:  pdflatex acks
